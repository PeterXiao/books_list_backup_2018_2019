\documentclass[compress]{beamer}%compress表示尽量压缩导航条
%\usetheme{Berlin}
\usepackage{thubeamer}
\usepackage{pgf}
\usepackage[UTF8,noindent]{ctex}%取消首行缩进

\usepackage{times} %设置文档中的所有英文为Times new roman字体
\usefonttheme{professionalfonts}    % 设置公式中的数学字体(wenzhong) http://bbs.ctex.org/forum.php?mod=viewthread&tid=76917

\setCJKsansfont[ItalicFont={KaiTi}]{SimSun}
%\logo{\includegraphics[width=1.3cm,height=1.3cm]{logo.pdf}} %在每个页面的右下角插入logo
\usepackage{caption}
\setbeamertemplate{caption}[numbered]{}% Number float-like environments

\begin{document}

\graphicspath{{figures/}} % figures path
\captionsetup[figure]{font=footnotesize,labelfont=footnotesize}

\title{基于大涡模拟的多旋流燃烧室\\两相燃烧流场研究}
\author[广末凉子]{学生:广末凉子\\ \vskip 5pt 导师:内田有纪}
\institute[南京航空航天大学]{\small \vskip 38pt 南京航空航天大学}
\date{\small \vskip -22pt \today}
%\titlegraphic{\includegraphics{logo.pdf}}
%\frame{
\begin{frame}
	\vspace{-10mm}
		\maketitle
	\vspace{-44mm}
	\begin{figure}[htbp]
		\begin{center}
			\includegraphics[width=0.14\linewidth]{logo.pdf}
		\end{center}
	\end{figure}
\end{frame}
\section*{目录}
\begin{frame}
	\frametitle{\secname}
	\tableofcontents[sections={<1-5>}]
\end{frame}
  %\AtBeginSubsection[] {
  %\frame<handout:0> {
  %\frametitle{目录}
  %\tableofcontents[current,currentsubsection,sections={<1-5>}]
    %}
    %\addtocounter{framenumber}{-1}  %目录页不计算页码
  %}
  
\AtBeginSection[] {%在每一节前面加入目录显示当前节的目录结构
	\begin{frame}
		\frametitle{目录}
		\tableofcontents[currentsection,currentsubsection,hideothersubsections,sectionstyle=show/shaded,]
		\addtocounter{framenumber}{-1}  %目录页不计入页码
	\end{frame}
}



\section{研究背景}
\begin{frame}
	\frametitle{\secname~ }
	\begin{block}{液体燃料燃烧}
		液体燃料在工业与民航中有着广泛的应用。
	\end{block}
	\begin{block}{低污染燃烧技术}
		国际民航组织对航空发动机排放的要求越来越高,研究低污染燃烧技术是促进我国民航事业发展的关键。
	\end{block}
	\begin{block}{数值模拟}
		计算流体力学的发展和广泛应用。
	\end{block}
\end{frame}
\section{研究内容}
\section{湍流两相燃烧的数学模型}
\begin{frame}{亚网格湍流模型}
  \begin{block}{Smagorinsky-Lilly模型}
    基于涡粘假设
  \end{block}
  \begin{block}{动态亚网格模型}
    通过可解尺度涡旋的局部特性来确定小尺度涡旋的模型系数
  \end{block}
\end{frame}
\begin{frame}{湍流燃烧模型}
稳态火焰面模型
\end{frame}
\begin{frame}
	\frametitle{液相亚网格随机模型}
	\begin{block}{运动方程}
		$$dx_p=U_pdt$$
	\[dU_p=\tau_p^{-1}( {\overline U}_g-U_p)dt+(C_0\frac{k_{sgs}}{\tau_t})^{1/2}dW_t\]
	\end{block}
\end{frame}
\section{数值求解方法}
\section{结果分析}
\section{Q\&A}
\begin{frame}{\secname~ }
	\begin{center}
 		\huge {\kaishu 结束,谢谢}
		
		\huge \textit {Q\&A}
	\end{center}
\end{frame}
\end{document}
